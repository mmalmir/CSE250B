\listfiles
\documentclass[twoside,12pt]{article}
\newcommand{\dataset}{{\cal D}}
\newcommand{\fracpartial}[2]{\frac{\partial #1}{\partial  #2}}
\usepackage{hyperref}
\usepackage{enumerate}
\usepackage{indentfirst}
\usepackage[top=2in, bottom=1.5in, left=0.85in, right=0.5in]{geometry}
\usepackage[hyphenbreaks]{breakurl}
%\usepackage[pdfstartview=FitH,pdfstartpage=13,pdfpagemode=UseNone]{hyperref}
\usepackage{amsfonts}
\usepackage{graphicx} 
\usepackage[linesnumbered,ruled]{algorithm2e}
\usepackage{float}
\usepackage{amssymb,amsmath}
\usepackage{mdwlist }
\usepackage{color}
\usepackage{graphics}
\definecolor{darkblue}{rgb}{0.0,0.0,0.5}
\newtheorem{Dfn}{Definition}
\usepackage{geometry}                % See geometry.pdf to learn the layout options. There are lots.
\geometry{letterpaper}   
\usepackage{epstopdf}
\epstopdfDeclareGraphicsRule{.tif}{png}{.png}{%
convert #1 \OutputFile
}
\AppendGraphicsExtensions{.tif}\usepackage{epstopdf}
\DeclareGraphicsRule{.tif}{png}{.png}{`convert #1 `dirname #1`/`basename #1 .tif`.png}

\hypersetup{colorlinks,breaklinks,
            linkcolor=darkblue,urlcolor=darkblue,
            anchorcolor=darkblue,citecolor=darkblue}
\newcommand{\sign}{\text{sign}}
\setlength{\parindent}{1cm}


\begin{document}
 \title{ Single Sideband Generation }
 \author{Class: Digital Signal Processing \\* \\* Erfan Sayyari \\* \\* PID: A53059770}
 \date{\today}
\maketitle
\newpage

\section{Text}

\subsection{Objective}

In this assignment we want to implement an FIR \emph{ Hilbert Transform} filter and generate \emph{single sideband} signals. Generally, in communication whenever we want to transmit a real signal using a cosine carrier, we are wasting bandwidth. So, in order to use the bandwidth more efficiently we may use single sideband form of the signals. In order to generate single sideband form of the signals we use FIR \emph{Hilbert Transform } filter. Since the FIR \emph{ Hilbert Transform} filter is \emph{antisymmetric} and desired length of the filter is 64 (even) it is type \textrm{IV} FIR filter.



Actually, by using \emph{Hilbert Transform} we are producing a complex signal whose imaginary part is produced by utilizing the \emph{Hilbert Transform}. In order to compute this complex signal we could simply compute the FFT of the signal, and make the negative part zero. However, in this implementation we do not use this method, and we will use an FIR \emph{Hilbert Transform} filter instead.   

\subsection{Introduction [1]}
In Digital Signal Processing (DSP) \emph{Hilbert Transform} filters are used to produce a different signal from input signal which has a $ \frac{\pi}{2} $ radians phase shift. The characteristic function of an ideal \emph{Hilbert Transform} is: 
\begin{equation*}
H(\omega)= \left\{
 \begin{array}{rl}
  +j & \text{if } -\pi<\omega < 0\\
   0 & \text{if } \omega = 0, -\pi\\
   -j & \text{if } 0<\omega<\pi
 \end{array} \right.
\end{equation*}

Although this system is not causal, we could still approximate it using FIR or IIR digital filters. Whenever we are using FIR \emph{Hilbert Transformers} we have to use suitable windowing, in order to reduce ripples and other bad side effects of hard filtering. One of the best window functions which has good frequency resolution is \emph{Hamming} window. The \emph{Min. Stopband Attenuation} of this window is -53 dB. This number shows maximum magnitude of the filter in stopband area in dB.

\subsection{Approach}
This assignment is built of 4 parts. In part A, we have to implement the block diagram which is given in Figure 11.11a in \cite{Aopen}. In order to do this, we have to implement FIR \emph{Hilbert Transform} filter first. In order to generate a 64 point FIR \emph{Hilbert Transform} filter we have used the command \emph{firpm(63, [ 0.1 0.9], [ 1 1 ], 'hilbert')} in MATLAB. The minimum frequency of the filter is come from $ 2*min{(f_1,f_2,f_3)} = 2*0.05$. The upper bound has been set symmetrically in order to have a low ripple FIR \emph{Hilbert Transform} filter. In order to generate single sideband signal, we need to generate a complex signal which is zero for $-\pi \leq \omega <0$:
\begin{equation*}
X(e^{j\omega})= 0,  \pi\leq\omega<0
\end{equation*}
Since the  \emph{Fourier} transform of the signal is not symmetric, it is not a real signal. We could write the general form of the signal as: 
\begin{equation*}
x[n]= x_r[n]+j*x_i[n] 
\end{equation*}
Where $ x_r[n]$ and $x_i[n]$ are real signals. The \emph{Fourier} transform of $x_r[n]$ and $x_i[n]$ are related to each other by the equation:
\begin{equation*}
X_i(e^{j\omega})= \left\{
 \begin{array}{rl}
  -jX_r(e^{j\omega}) & \text{if } 0<\omega < \pi\\
 +jX_r(e^{j\omega}) & \text{if } -\pi\leq \omega <0 \\
 \end{array} \right.
 \end{equation*}
  Or we could write:
  \begin{equation*}
  X_i(e^{j\omega})=H(e^{j\omega})X_r(e^{j\omega}),
  \end{equation*}
  Where:
  
  \begin{equation*}
H(e^{j\omega})= \left\{
 \begin{array}{rl}
  -j & \text{if } 0<\omega < \pi\\
 +j & \text{if } -\pi\leq \omega <0 \\
 \end{array} \right.
\end{equation*}.

Hence, in this assignment we have the real part of the signal, $x_r[n]=a_1sin(2\pi f_1n)+a_2sin(2\pi f_2n)+a_3sin(2\pi f_3n)$. Using the \emph{Hilbert Transform} (which is produced by firpm command of MATLAB), we can generate the imaginary part of the signal. The \emph{Hilbert Transform} of $a*sin(2\pi f*n)$ is $a*cos(2\pi fn)$. 

We have implemented \emph{Hilbert Transform} by a 64 FIR filter. This system needs synched $x_r[n] and x_i[n]$ to work properly. The FIR \emph{Hilbert Transform} filter has a linear delay. So in order to have a synched real part of signal before multiplying by carriers, we need to have a delayed form of $x_r[n]$. In order to implement delayed form of $x_r[n]$, we will use a 64 FIR lowpass filter, which is one over the working frequency band, $max{(f_1,f_2,f_3)}$. 

In this block diagram, we will multiply $x_r[n]$ by $cos(\omega_cn)$ to get $y_r[n]$, and we will multiply $x_i[n]$ by $sin(\omega_cn)$ to get $y_i[n]$. If we add $y_r[n] and y_i[n]$ we have LSB, and if we subtract $y_i[n[$ from $y_r[n]$ we will get USB. In order to show this we could consider $x_r=a*sin(2\pi fn)$. So we have:

 \begin{equation*}
x_r[n]=a*sin(2\pi fn)
\end{equation*}
 \begin{equation*}
x_i[n]=a*cos(2\pi fn)
\end{equation*}
\begin{equation*}
y_r[n]=a*sin(2\pi fn)*cos(\omega_cn)
\end{equation*}

\begin{equation*}
y_i[n]=a*cos(2\pi fn)*sin(\omega_cn).
\end{equation*}

So for LSB we have:
 \begin{equation*}
y[n] = y_r[n]+y_i[n]
\end{equation*}
or
\begin{equation*}
y[n]=a*sin(2\pi fn)*cos(2\omega_cn)+a*cos(2\pi fn)*sin(\omega_cn)
\end{equation*}
\begin{equation*}
y[n]=a*sin((2\pi fn+\omega_cn)n).
\end{equation*}

It is obvious that we have one side of the signal. It is more clear if we look at the case where we want to transmit just $x_r=a*sin(2\pi fn)$. In this case if we multiply it by the carrier $cos(\omega_cn)$ we will have 4 delta functions. 

Consequently, we can do the same procedure to get the USB:
\begin{equation*}
y[n] = y_r[n]-y_i[n]
\end{equation*}
or
\begin{equation*}
y[n]=a*sin(2\pi fn)*cos(2\omega_cn)-a*cos(2\pi fn)*sin(\omega_cn),
\end{equation*}
\begin{equation*}
y[n]=a*sin((2\pi fn-\omega_cn)n).
\end{equation*}

Clearly, this procedure is true for arbitrary signals, since we could decompose them as the sum of some sine or cosine signals. 

In part B, we have a 1024 point signal, $x_r[n]$ which is consist of three low frequency sinusoids ($f_1 = 0.05, f_2 = 0.075, and f_3 = 0.10 cycles/sample$). They have 10 dB difference in magnitude with each other.  The carrier frequency is $f_c = 0.25 cycles/sample$. And we design a 64 FIR \emph{Hilbert Transform} filter  and 64 FIR lowpass filter using \emph{firpm} as mentioned above.

In part C, we have plotted the impulse response and transform function of both lowpass and \emph{Hilbert} filter.  

In part D, we have plotted the magnitude (in dB) of FFTs at each stage of the system from input through output including FFTs of $x_r[n], x_r�[n]$  (lowpass filtered form of $x_r[n]$), $x_i(n)$, $ x_r�(n) cos(2\pi f_cn)$, $ x_i(n) sin(2\pi fc_n)$, and $s_r(n)$ (both USB and LSB). Also, we have plotted FFTs of $x(n) = x_r�(n) + j x_i(n)$ and $s(n) = x(n) exp(+j2\pi f_cn)$. To compute FFTs, NFFT = 256 and \emph{Hamming} window is used. 
 
\subsection{Results}
We have described the successful generation of both upper and lower single sideband signals in previous section. 

The impulse response and transfer function of \emph{Hilbert} filter are plotted in Fig 1. As it is obvious, the impulse response is antisymmetric and the magnitude of the FFT of the designed \emph{Hilbert Transform} filter is zero (dB) over the range [0.1 0.9] (cycles/samples). The phase of designed \emph{Hilbert Transform} is $\frac{-\pi}{2}$ for the $0< \omega< \pi$ and $\frac{\pi}{2}$ for the $-\pi<\omega<0$.

Furthermore, the magnitude and phase of lowpass filter are plotted in Fig 2,3 and 4. As it is visible the magnitude of the lowpass filter is zero (in dB) over our working range. In contrast with impulse response of \emph{Hilbert Transform} filter, the impulse response of lowpass filter is symmetric.

The FFT (dB magnitude) of $x_r[n], x_r'[n]$ and $x_i[n]$ are plotted in Figs 5 and 6. They are symmetric around 0.5, and the impulses have 10 dB differences, as the magnitude of original signals have. The magnitude of FFTs of $x_r�[n] cos(2\pi f_cn)$ and  $x_i(n) sin(2\pi f_cn)$ are plotted in Figs 7. This figures show that, not only they are symmetric around 0.5, they are symmetric around 0.25 as well, which is redundant. If we add them up or subtract them to get $s_r[n]$ (or y[n] for above notation) to have Fig 8, we could see that we do not have the redundant symmetry anymore. Besides, for the LSB we could see that the frequency of the signal is around $f _1'= f_1+f_c=-0.05+0.25=0.2$, $f _2'= f_2+f_c=-0.075+0.25=0.175$ and $f _3'= f_3+f_c=-0.1+0.25=0.15$. For the USB we could see that the frequency of the signal is around $f _1'= f_1-f_c=-0.05-0.25=0.3$, $f _2'= f_2-f_c=0.325$ and $f _3'= f_3-f_c=0.35$. We have plotted $x[n]=x_r'[n]+jx_i[n]$ and $s[n]=x[n]e^{j2\pi f_cn}$ in Fig 9. As it is visible they are not symmetric anymore. And for $\omega<0$, $X(e^{j\omega})$ is zero.  And it is visible that this signal is still analytical.
\subsection{Conclusions}

In this assignment we have implemented a block diagram of single sideband signal generator. In the case of transmitting a signal using a sine carrier, we will use the bandwidth more than it is necessary. By using the \emph{Hilbert Transform} we could get rid of this redundancy. \emph{Hilbert Transform} is antisymmetric. Furthermore, By using this transform we could produce an analytical form of signal. And if $f_1+f_c<0.5$ the single side band form of signal is still analytical.

\end{document}